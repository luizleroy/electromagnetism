Qual a melhor ordem para determinar os espaços nulo, linha e coluna?
\\Se a questão é resolver para uma mesma matriz estes três problemas é sempre mais vantajoso, ou seja, é gerado menos cálculos, se executarmos os seguintes passos:
\item Encontrar a matriz escalonada por linhas. Dela obtemos o espaço linha.
\item Encontrar a matriz escalonada reduzida. Dela, com auxílio dos pivôs, obtemos o espaço coluna.
\item Com a matriz escalonada reduzida o sistema linear poderá ser resolvido e, assim, encontramos o espaço nulo.

Quais teoremas já foram necessários enunciar em anos anteriores?
\\Seguem os teoremas e um ponteiro para as demostrações:
\item O teorema espectral para operadores auto-adjuntos. \cite{x}
\item O teorema do núcleo e da imagem. \cite{y}
\item A desigualdade de Bessel
\item Três condições equivalentes para que uma matriz $n\times n$ seja invertível.
\item Identidade de Jacobi
\item Teorema de Gren-Shidt \cite{406}

O que é decomposição espectral?

Como diagonalizar uma matriz?

O que é reduzir a uma forma quadrática uma determinada expressão?

O que é identidade de Jacobi?

Como resumir ao máximo a aplicação do teorema de Gren-Schmidt? \cite{406}

Como diagonalizar uma matriz? \cite{158}




