\documentclass[a4paper]{article}

\usepackage[english]{babel}
\usepackage[utf8]{inputenc}
\usepackage{amsmath}
\usepackage{graphicx}
\usepackage[colorinlistoftodos]{todonotes}
\usepackage{enumitem}% http://ctan.org/pkg/enumitem

\newtheorem{theorem}{Theorem}[section]
\newtheorem{lemma}[theorem]{Lemma}
\newtheorem{proposition}[theorem]{Proposition}
\newtheorem{corollary}[theorem]{Corollary}
\newenvironment{proof}[1][Proof]{\begin{trivlist}
\item[\hskip \labelsep {\bfseries #1}]}{\end{trivlist}}
\newenvironment{definition}[1][Definition]{\begin{trivlist}
\item[\hskip \labelsep {\bfseries #1}]}{\end{trivlist}}
\newenvironment{example}[1][Example]{\begin{trivlist}
\item[\hskip \labelsep {\bfseries #1}]}{\end{trivlist}}
\newenvironment{remark}[1][Remark]{\begin{trivlist}
\item[\hskip \labelsep {\bfseries #1}]}{\end{trivlist}}
\newcommand{\qed}{\nobreak \ifvmode \relax \else
      \ifdim\lastskip<1.5em \hskip-\lastskip
      \hskip1.5em plus0em minus0.5em \fi \nobreak
      \vrule height0.75em width0.5em depth0.25em\fi}

\title{How to Write Mathematical Objects}
\author{Luiz Le Roy}
\date{\today}

\begin{document}
\maketitle

\begin{abstract}

\end{abstract}

\section{English}
\begin{definition}
Let $H$ be a subgroup of a group~$G$.  A \emph{left coset} of $H$ in $G$ is a subset of $G$ that is of the form $xH$, where $x \in G$ and $xH = \{ xh : h \in H \}$. Similarly a \emph{right coset} of $H$ in $G$ is a subset of $G$ that is of the form $Hx$, where $Hx = \{ hx : h \in H \}$
\end{definition}

Note that a subgroup~$H$ of a group $G$ is itself a left coset of $H$ in $G$.

\begin{lemma}
\label{LeftCosetsDisjoint}
Let $H$ be a subgroup of a group $G$, and let $x$ and $y$ be elements of $G$.  Suppose that $xH \cap yH$ is non-empty. Then $xH = yH$.
\end{lemma}

\begin{proof}
Let $z$ be some element of $xH \cap yH$.  Then $z = xa$ for some $a \in H$, and $z = yb$ for some $b \in H$. If $h$ is any element of $H$ then $ah \in H$ and $a^{-1}h \in H$, since $H$ is a subgroup of $G$. But $zh = x(ah)$ and $xh = z(a^{-1}h)$ for all $h \in H$. Therefore $zH \subset xH$ and $xH \subset zH$, and thus $xH = zH$.  Similarly $yH = zH$, and thus $xH = yH$, as required.\qed
\end{proof}

\begin{lemma}
\label{SizeOfLeftCoset}
Let $H$ be a finite subgroup of a group $G$.  Then each left coset of $H$ in $G$ has the same number of elements as $H$.
\end{lemma}

\begin{proof}
Let $H = \{ h_1, h_2,\ldots, h_m\}$, where $h_1, h_2,\ldots, h_m$ are distinct, and let $x$ be an element of $G$.  Then the left coset $xH$ consists of the elements $x h_j$ for $j = 1,2,\ldots,m$. Suppose that $j$ and $k$ are integers between $1$ and $m$ for which $x h_j = x h_k$.  Then $h_j = x^{-1} (x h_j) = x^{-1} (x h_k) = h_k$, and thus $j = k$, since $h_1, h_2,\ldots, h_m$ are distinct.  It follows that the elements $x h_1, x h_2,\ldots, x h_m$ are distinct. We conclude that the subgroup~$H$ and the left coset $xH$ both have $m$ elements, as required.\qed
\end{proof}

\begin{theorem}
\emph{(Lagrange's Theorem)}
\label{Lagrange}
Let $G$ be a finite group, and let $H$ be a subgroup of $G$.  Then the order of $H$ divides the order of $G$.
\end{theorem}

\begin{proof}
Each element~$x$ of $G$ belongs to at least one left coset of $H$ in $G$ (namely the coset $xH$), and no element can belong to two distinct left cosets of $H$ in $G$ (see Lemma~\ref{LeftCosetsDisjoint}).  Therefore every element of $G$ belongs to exactly one left coset of $H$. Moreover each left coset of $H$ contains $|H|$ elements (Lemma~\ref{SizeOfLeftCoset}).  Therefore $|G| = n |H|$, where $n$ is the number of left cosets of $H$ in $G$. The result follows.\qed
\end{proof}


\end{document}
