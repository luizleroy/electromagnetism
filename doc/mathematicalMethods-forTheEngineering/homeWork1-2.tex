\documentclass[11pt]{article}
\usepackage[utf8]{inputenc}
\usepackage{amsmath} %(?)
\usepackage{amsthm}%for theorem  
\usepackage{hyperref}
\usepackage{amssymb} % In LaTeX, how do I represent the hollow "R" symbol that designates the real number space? -> use \mathbb from amsfonts (loaded by amssymb)
\title{Math by \LaTeX}
\date{}
\author{Luiz Le Roy}
\begin{document}
  \maketitle
  \today
  
\begin{center}
{\LARGE \textbf{  Primeiro conjunto de tarefas}}
\end{center}

\begin{itemize}
\item[1] Um exemplo de matriz $A \neq 0$ tal que $A^2=0$:
$$
\left[\begin{array}{rr}
a&b\\
c&d
\end{array}\right]
\quad
\left[\begin{array}{cc}
a&b\\
c&d
\end{array}\right]
=
\quad
\left[\begin{array}{cc}
0&0\\
0&0
\end{array}\right]
$$
\textbf{Onde a = 1, b = 1, c = -1 e d = -1.}
\item[2] As matrizes quadradas de ordem n, A e B, comutam se $AB = BA$. 
\\As matrizes 2x2 que comutam com $\left[\begin{array}{cc}
1&0\\0&2
\end{array}\right]$ são:
$$
\begin{bmatrix}
1&0\\0&2
\end{bmatrix}
\begin{bmatrix}
a&b\\c&d
\end{bmatrix}
=
\begin{bmatrix}
a&b\\c&d
\end{bmatrix}
\begin{bmatrix}
1&0\\0&2
\end{bmatrix}
$$
\textbf{Com a = 0, c = 0, b = x e d = y onde $x,y\hspace{2mm}\in\hspace{2mm}\mathbb{R}^2$.}
\\As matrizes 3x3 que comutam com 
$\begin{bmatrix}
1&1&0\\1&1&1\\0&0&1
\end{bmatrix}$ são:
$$
\begin{bmatrix}
1&1&0\\1&1&1\\0&0&1
\end{bmatrix}
\begin{bmatrix}
a&b&c\\d&e&f\\g&h&i
\end{bmatrix}
=
\begin{bmatrix}
a&b&c\\d&e&f\\g&h&i
\end{bmatrix}
\begin{bmatrix}
1&1&0\\1&1&1\\0&0&1
\end{bmatrix}
$$
\textbf{Com b = f = a, i = e, d = 0, g = 0 e h = 0 onde $a,c,e\hspace{2mm}\in\hspace{2mm}\mathbb{R}^3$.}

\item[3] Com auxílio do seguinte teorema:
\\ Se A é uma matriz quadrada e se existe um inteiro positivo k tal que $A^k=0$, então a matriz $I-A$ é invertível e $$(I-A)^{-1} = I + A + A^2 + ... + A^{k-1}$$
É possível aplicar facilmente para o caso em questão, onde $k=3$.
\begin{proof}
Por exemplo, se x é um número real qualquer e k é um inteiro positivo, então a identidade algébrica $$(1-x)(1+x+x^2+...+x^{k-1}$$ traduz para a seguinte identidade entre matrizes quadradas:
$$
(I-A)(I+A+A^2+...+A^{k-1})=I-A^k
$$
Isso pode ser confirmado multiplicando os fatores do lado esquerdo, pois temos o que se chada de soma telescópica (a origem de \textit{Telescoping series}):
$$
(I+A+A^2+...+A^{k-1}) - (A+A^2+A^3...+A^k) = I-A^k
$$
Com auxílio das equações acima pode-se concluir que no caso em questão, onde $k=3$ e $A^k=0$,      $I-A$ possui inversa (da definição) e seu valor é $I + A + A^2$ pois qualquer potência maior que três também vale zero quando $A^3=0$.\qedhere
\end{proof}

%%
%% itens não finalizados
%%

\item[4] Sejam A e B nxn tais que $A=A^t$, $B=B^t$, isto é, simétricas. Prove que AB é simétrica se, e só se, A e B comutam.

\item[5] Uma matriz de probabilidade é uma matriz $P=(p_{ij})_{n \times n}$ tal que $p_{ij}\geq0$ pra todo par (i,j) e $\sum_{j=1}^{n}(p_{ij})=1$ para todo i. Prove que se P e Q são matrizes de probabilidade $n \times n$, então PQ também o é.

\item[6] Resolva os sistemas:\\
\begin{itemize}
\item[a]
$ \left\{
\begin{array}{ll}
\displaystyle 2x_1 - 3x_2 + 7x_3 = 2 \\
\displaystyle 6x_1 - 9x_2 + 21x_3 = -1
\end{array}
\right.
$
\item[b]
$ \left\{
\begin{array}{ll}
\displaystyle 6x_1+9x_2+14x_3=0 \\
\displaystyle 3x_1-8x_2+7x_3=0
\end{array}
\right.
$
\item[c]
$ \left\{
\begin{array}{llll}
\displaystyle x_1+x_2-x_3=-2 \\
\displaystyle 2x_1+x_2-3x_3=-7 \\
\displaystyle 4x_1-5x_2-13x_3=-35 \\
\displaystyle -3x_1+2x_2+8x_3=21
\end{array}
\right.
$
\item[d]
$ \left\{
\begin{array}{llll}
\displaystyle 2x_1+x_2-3x_3=0 \\
\displaystyle x_1-x_3=0 \\
\displaystyle -5x_1+x_2+2x_3=0 \\
\displaystyle 12x_1-x_2-x_3=0
\end{array}
\right.
$
\end{itemize}

\item[7] Calcule a e b de modo que sejam compatíveis com as equações:\\
$ \left\{
\begin{array}{llll}
\displaystyle 4x_1+3x_2=-3 \\
\displaystyle 5x_1+4x_2=-5 \\
\displaystyle ax_1+2x_2=5 \\
\displaystyle 3x_1+bx_2=-1
\end{array}
\right.
$

\item[8] Discuta como o posto de A varia com teR, onde:

\end{itemize}

\begin{center}
{\LARGE \textbf{Segundo conjunto de tarefas}}
\end{center}

\begin{itemize}
\item[1] Sejam $A_{m \times n}$, $P_{m \times m}$ invertível, $\textbf{a}_1, \textbf{a}_2, ..., \textbf{a}_r$ r colunas L.I de A. Prove que $P\textbf{a}_1,P\textbf{a}_2,...,P\textbf{a}_r$ são r colunas L.I. de PA.
\begin{proof}
É preciso da seguinte equação:
\[
a^2 + b^2 = c^2
\]
\textit{Thats all folks!} \qedhere
\end{proof}

\item[2] Discuta as soluções do sistema:
\begin{eqnarray*} 
ax \hspace{9mm} +bz &=&1\\
ax+ay+4z&=&2\\
ay+2z&=&3 
\end{eqnarray*}
\end{itemize}

%
%
%Seja A_{nxn} tal que A^3=0. Prove que (I_n - A)^{-1}=I_n + A + A^2.
%
%Sejam A e B nxn tais que A=A^t, B=B^t, isto �, sim�tricas. Prove que AB � sim�trica se, e s�se, A e B comutam.
%
%Uma matriz de probabilidade � uma matriz P=(p_{ij}) nxn tal que p_{ij}>=0 para todo par (i,j) e \sum{pij}=1 para todo i. Prove que se P e Q s�o matrizes de probabilidade nxn, ent�o PQ tamb�m o �.
%
%Resolva os sistemas:
%2x1 - 3x2 + 7x3 = 2
%6x1 -9x2+21x3=-1
%
%6x1+9x2+14x3=0
%3x1-8x2+7x3=0
%
%x1+x2-x3=-2
%2x1+x2-x3=-7
%4x1-5x2-13x3=-35
%-3x1+2x2+8x3=21
%
%2x1+x2=x3=0
%x1 - x3=0
%-5x1+x2+2x3=0
%12x1-x2-x3=0
%
%Calcule a e b de modo que sejam compat�veis os equa��es:
%4x1+3x2=-3
%5x1+4x2=-5
%ax1+2x2=5
%3x1+bx2=-1
%
%Discuta como o posto de A varia com teR, onde A = [1 1 t; 1 t 1; t 1 1].
%
%Discuta as solu��es do sistema {x + ay =1
%								ax+y=a^3} conforme os valores do par�metro a e R.
%
%D� o exemplo de sistema linear com 4 equa��es e 4 inc�gnitas cuja solu��o dependa de:
%a) 1 par�metro, b) 2 c) 3 
								
\end{document}

%%exemplo de sistema
%\begin{eqnarray*} 
%a \hspace{18mm} &=&1\\
%\hspace{9mm}b\hspace{9mm}&=&2\\
%c&=&3
%\end{eqnarray*}

