\documentclass[11pt]{article}
\usepackage[utf8]{inputenc}
\usepackage{amsmath}
\usepackage{hyperref}
\title{\LaTeX}
\date{}
\begin{document}
  \maketitle

\begin{itemize}
\item Um exemplo de matriz $A \neq 0$ tal que $A^2=0$:
$$
\left[\begin{array}{rr}
a&b\\
c&d
\end{array}\right]
\quad
\left[\begin{array}{cc}
a&b\\
c&d
\end{array}\right]
=
\quad
\left[\begin{array}{cc}
0&0\\
0&0
\end{array}\right]
$$
Onde $a=1$, $b=1$, $c=-1$ e $d=-1$.
\item As matrizes quadradas de ordem n, A e B, comutam se $AB \neq BA$. Ache as matrizes 2x2 que comutam com [1 0; 0 2]. Ache as matrizes 3x3 que comutam com [1 1 0; 0 1 1; 0 0 1].
\\ Será este mais bonito: ovo mutum amo mutum ovo?
\end{itemize}



%
%
%Seja A_{nxn} tal que A^3=0. Prove que (I_n - A)^{-1}=I_n + A + A^2.
%
%Sejam A e B nxn tais que A=A^t, B=B^t, isto �, sim�tricas. Prove que AB � sim�trica se, e s�se, A e B comutam.
%
%Uma matriz de probabilidade � uma matriz P=(p_{ij}) nxn tal que p_{ij}>=0 para todo par (i,j) e \sum{pij}=1 para todo i. Prove que se P e Q s�o matrizes de probabilidade nxn, ent�o PQ tamb�m o �.
%
%Resolva os sistemas:
%2x1 - 3x2 + 7x3 = 2
%6x1 -9x2+21x3=-1
%
%6x1+9x2+14x3=0
%3x1-8x2+7x3=0
%
%x1+x2-x3=-2
%2x1+x2-x3=-7
%4x1-5x2-13x3=-35
%-3x1+2x2+8x3=21
%
%2x1+x2=x3=0
%x1 - x3=0
%-5x1+x2+2x3=0
%12x1-x2-x3=0
%
%Calcule a e b de modo que sejam compat�veis os equa��es:
%4x1+3x2=-3
%5x1+4x2=-5
%ax1+2x2=5
%3x1+bx2=-1
%
%Discuta como o posto de A varia com teR, onde A = [1 1 t; 1 t 1; t 1 1].
%
%Discuta as solu��es do sistema {x + ay =1
%								ax+y=a^3} conforme os valores do par�metro a e R.
%
%D� o exemplo de sistema linear com 4 equa��es e 4 inc�gnitas cuja solu��o dependa de:
%a) 1 par�metro, b) 2 c) 3 
								
\end{document}
