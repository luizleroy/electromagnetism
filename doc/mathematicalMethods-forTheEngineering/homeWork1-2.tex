\documentclass[a4paper]{article}
\usepackage[utf8]{inputenc}
\usepackage{amsmath} %(?)
\usepackage{hyperref}
\usepackage{color}
\usepackage{amssymb} % In LaTeX, how do I represent the hollow "R" symbol that designates the real number space? -> use \mathbb from amsfonts (loaded by amssymb)
%\title{Math by \LaTeX}
\usepackage{datetime}
%\newdateformat{mydate}{\monthname[\THEMONTH] \THEYEAR}
\newdateformat{mydate}{\THEDAY /\THEMONTH /\THEYEAR}


%\usepackage[backend=bibtex]{biblatex} %to cite
%\bibliography{myalgebra} 

\newtheorem{theorem}{Theorem}[section]
\newtheorem{lemma}[theorem]{Lemma}
\newtheorem{proposition}[theorem]{Proposition}
\newtheorem{corollary}[theorem]{Corollary}
\newenvironment{proof}[1][Proof]{\begin{trivlist}
\item[\hskip \labelsep {\bfseries #1}]}{\end{trivlist}}
\newenvironment{definition}[1][Definition]{\begin{trivlist}
\item[\hskip \labelsep {\bfseries #1}]}{\end{trivlist}}
\newenvironment{example}[1][Example]{\begin{trivlist}
\item[\hskip \labelsep {\bfseries #1}]}{\end{trivlist}}
\newenvironment{remark}[1][Remark]{\begin{trivlist}
\item[\hskip \labelsep {\bfseries #1}]}{\end{trivlist}}
\newcommand{\qed}{\nobreak \ifvmode \relax \else
      \ifdim\lastskip<1.5em \hskip-\lastskip
      \hskip1.5em plus0em minus0.5em \fi \nobreak
      \vrule height0.75em width0.5em depth0.25em\fi}

\title{Exercicios da Primeira Etapa - Prova I}
\date{\mydate\today}
\author{Luiz Le Roy}
\begin{document}
  \maketitle

\section*{Lista 3: Espaços Vetoriais Arbitrários}
\begin{itemize}
\item[1] O conjunto das funções $f:\mathbb{R}\longrightarrow \mathbb{R}$ duas vezes continuamente deriváveis e tais que $f''+af'+bf=0$, onde $\underline{a}$ e $\underline{b}$ são reais fixos, munido das leis usuais, é um espaço vetorial?
\\\textcolor[rgb]{0,0,1}{Pelas leis usuais do cálculo a função identicamente nula encontra-se no conjunto acima. Temos também, do cálculo, a garantia de que a soma de duas funções do espaço acima e o produto por um escalar, também satisfazem a equação portanto pertencem ao espaço descrito. É possível observar também que as funções duas vezes continuamente deriváveis são um subespaço de $F(\infty,\infty)$ .%conforme  \cite{anton2013elementary} %541-543
}
\item[2] Prove que o conjunto das funções limitadas $f:\mathbb{R}\longrightarrow \mathbb{R}$ (isto é, tais que existe $M_f>0$ tal que $|f(x)|\leq M_f$ para todo $x\in \mathbb{R}$), com as leis usuais, é um espaço vetorial.
\\\textcolor[rgb]{0,0,1}{É fácil mostrar que, este tipo de função, é um subespaço de $f:\mathbb{R}\longrightarrow \mathbb{R}$. Por ser um conjunto fechado na adição e multiplicação por um escalar. Portanto ele é um espaço vetorial.}

\item[3] Seja $V=\mathfrak{F}(\mathbb{R,R})$. Verifique se W é subespaço de V onde:
\begin{itemize}
\item [a)] W é $f(-x)=f(x)$
\\\textcolor[rgb]{0,0,1}{É subespaço, pois $f(-x)+g(-x)=f(x)+g(x)$ e $af(-x)=af(x)$.}
\item [b)] W é $f(-x)=-f(x)$
\\\textcolor[rgb]{0,0,1}{É subespaço, pois $f(-x)+g(-x)=-f(x)-g(x)$ e $af(-x)=-af(x)$.}
\item[c)] W é o conjunto das funções deriváveis.
\\\textcolor[rgb]{0,0,1}{Sim. Pois, pelas leis usuais do cálculo, um conjunto de funções deriváveis é fechado na adição e na multiplicação por um escalar.}
\end{itemize}

\item[4] Mostre que $W=\{(0,y,z)\in \mathbb{R}^3;y,z \in \mathbb{R}\}$ é subespaço  de $\mathbb{R}^3$ gerado por (0,1,1) e (0,2,-1).
\\ \textcolor[rgb]{0,0,1}{
Observa-se que qualquer combinação dos vetores acima mantém o valor de $x=0$. Portanto basta analisar o seguinte sistema:
\\
$
\left[\begin{array}{rr}
1&2\\
1&-1
\end{array}\right]
\quad
\left[\begin{array}{c}
\alpha\\
\beta
\end{array}\right]
=
\quad
\left[\begin{array}{c}
y\\
z
\end{array}\right]
$
\\ Como o determinante da matriz é diferente de zero, a solução do sistema é única, mostrando que o subespaço é gerado pelos vetores.
}

\item[5] Mostre que $p(t)=t^3-t^2+1$, $q(t)=t^2-1$ e $r(t)=2t^3+t-1$ são L.I. em $P_4$.
\item[6] Prove que $f(t)=t$, $g(t)=e^t$ e $h(t)=sen(t)$ são L.I.
\item[7] Seja W o subespaço de $P_4$ gerado por $u=t^3-t^2+1$, $v=t^2-1$ e $w=t^3-3t^2+3$. Ache uma base de W.
\\\textcolor[rgb]{0,0,1}{A base abaixo, do espaço linha da matriz A é equivalente ao procurado:
$$
\left[\begin{array}{rrrr}
1&-1&0&1\\
0&1&0&-1\\
1&-3&0&3
\end{array}\right]
\rightarrow
\left[\begin{array}{rrrr}
1&-1&0&1\\
0&1&0&-1\\
0&2&0&-2
\end{array}\right]
\rightarrow
\left[\begin{array}{rrrr}
1&-1&0&1\\
0&1&0&-1\\
0&0&0&0
\end{array}\right]
$$
Após reduzir por linhas a matriz basta extrair a base a partir das linhas não nulas:
(1,-1,0,1) e (0,1,0,-1).}

\item[8] Existe base de $P_4$ que não contenha nenhum polinômio de grau 2?
\textcolor[rgb]{0,0,1}{\\Não. Para ser geradora do espaço é necessário todos os termos do polinômio.}
\item[9] Seja $dimV=n$. Prove que se $x={v_1,v_2,...,v^p}$ gera V, então X contém uma base de V.

\item[10] Prove que se $v_1,...,v_n$ geram V e $p>n$, então $u_1,...,u_p\in V$ são L.D.

\item[11] Dado o conjunto finito $X=\{a_1,a_2,...,a_n\}$, ache uma base para o espaço vetorial $V=\mathfrak{F}(X,\mathbb{R})=\{f:X\longrightarrow \mathbb{R}\}$.

\item[12] Ache $T:\mathbb{R}^2\longrightarrow \mathbb{R}$ tal que $T(1,1)=-1$ e $T(1,0)=3$, T linear.
\\
\textcolor[rgb]{0,0,1}{
Basta resolver o sistema:
$$
\left[\begin{array}{rr}
1&1\\
1&0
\end{array}\right]
\quad
\left[\begin{array}{c}
\alpha\\
\beta
\end{array}\right]
=
\quad
\left[\begin{array}{c}
-1\\
2
\end{array}\right]
$$
Onde $\alpha=3$  e $\beta=-4$.
}

\item[13] Ache $T:\mathbb{R}^3\longrightarrow \mathbb{R}^4$ linear tal que ImT seja gerada por (1,0,2,-4) e (0,2,-1,3).

\item[14] Seja $T:V\longrightarrow V$ linear. Prove que se $T(v_1),...,T(v_n)$ são L.I., então $v_1,...,v_n$ são L.I.

\item[15] Sejam L, $T:V\longrightarrow V$ isomorfismos. Prove: $(L\circ T)^{-1}=T^{-1}\circ L^{-1}$.
\end{itemize}

\section*{Lista 2: Determinantes e Matrizes. Também temos: Conjuntos L.I e L.D.}

\begin{itemize}
\item[1] Sejam $A_{m \times n}$, $P_{m \times m}$ invertível, $\textbf{a}_1, \textbf{a}_2, ..., \textbf{a}_r$ r colunas L.I de A. Prove que $P\textbf{a}_1,P\textbf{a}_2,...,P\textbf{a}_r$ são r colunas L.I. de PA.

\item[2] Discuta as soluções do sistema:
\begin{eqnarray*} 
ax \hspace{9mm} +bz &=&1\\
ax+ay+4z&=&2\\
ay+2z&=&3 
\end{eqnarray*}

\textcolor[rgb]{0,0,1}{Com auxílio da matriz aumentada, e operações como permutação e combinações de linhas temos a seguinte sequência, válida apenas para $a\neq0$ e $b\neq2$:
$$
\begin{bmatrix}
a&0&b&1\\0&a&2&3\\a&a&4&2
\end{bmatrix}
\begin{bmatrix}
1&0&\frac{b}{a}&\frac{1}{a}\\0&1&\frac{2}{a}&\frac{3}{a}\\0&a&4-b&1
\end{bmatrix}
\begin{bmatrix}
1&0&\frac{b}{a}&\frac{1}{a}\\0&1&\frac{2}{a}&\frac{3}{a}\\0&0&2-b&-2
\end{bmatrix}
\Rightarrow
$$
$$
\Rightarrow
\begin{bmatrix}
x\\y\\z
\end{bmatrix}
=
\begin{bmatrix}
\frac{1}{a}-\frac{3b}{a^2} + \frac{4b}{a^2(b-2)}\\\frac{3}{a}-\frac{4}{a(b-2)}\\\frac{2}{b-2}
\end{bmatrix}
$$
Se $a=0$ ou $b=2$ o sistema é impossível.}

\item[3] Sem calcular o determinante, prove que 
$ \left \vert
\begin{array}{lll}
\displaystyle \text{1 2 3} \\
\displaystyle \text{4 5 6} \\
\displaystyle \text{7 8 9}
\end{array}
\right \vert = 0.
$
\\ \textcolor[rgb]{0,0,1}{Basta observar que a linha dois é um múltiplo da linha 1. Neste caso o determinante é igual a 0.}
\item[4] Prove que
$ v(a,b,c) = \left \vert
\begin{array}{lll}
\displaystyle 1&1&1 \\
\displaystyle a&b&c \\
\displaystyle a^2&b^2&c^2
\end{array}
\right \vert = (c-a)(c-b)(b-a).
$
\\ \textcolor[rgb]{0,0,1}{
$
\left \vert
\begin{array}{lll}
\displaystyle 1&1&1 \\
\displaystyle a&b&c \\
\displaystyle a^2&b^2&c^2
\end{array}
\right \vert = \left \vert
\begin{array}{lll}
\displaystyle 1&1&1 \\
\displaystyle 0&b-a&c-a \\
\displaystyle 0&b^2-a^2&c^2-a^2
\end{array}
\right \vert = $
\\
$(b-a)(c^2-a^2)-(c-a)(b^2-a^2) = $
\\
$(b-a)(c-a)(c+a)-(c-a)(b-a)(b+a) = $
\\
$(b-a)(c-a)[(c+a)-(b+a)] = $
\\
$(c-a)(c-b)(b-a).
$
}

\item[5] Seja $A_{m\times n}$. Se $P_{n\times n}$ é invertível e $B=P^{-1}AP$, prove que $det(A)=det(B)$, e que $det(B-\lambda I_n)=det(A-\lambda I_n)$ qualquer que seja $\lambda \in \mathbb{R}$.
\\ \textcolor[rgb]{0,0,1}{
$det(B)=det(P^{-1}AP)=det(P^{-1})det(A)det(B)=\frac{1}{p}det(A)p=det(A)$
\\
$det(B-\lambda I_n)=det(B)-\lambda det(I_n)=det(A)-\lambda det(I_n)=det(A-\lambda I_n)$
}

\item[6] Sejam $\mathfrak{v}_1,...,\mathfrak{v}_n\text{ }\in \mathbb{R}^n$, $\mathfrak{v}_j=\sum_{i=1}^n a_{ij}e_i$, $A=(a_{ij})$, $\mathcal{E}=(e_1,...,e_n)$ a base canônica do $\mathbb{R}^n$. Prove que $\mathfrak{v}_1,...,\mathfrak{v}_n$ são L.I. se, e só se, $det(A)\neq0$.

\item[7] Uma matriz quadrada H é \underline{ortogonal} se $HH^t=I_n$. Prove que o produto de duas matrizes ortogonais é uma matriz ortogonal, e que $det(H)=\pm 1$.
\\ \textcolor[rgb]{0,0,1}{
Temos: $(AB)(AB)^t=ABB^tA^t=AIA^t=AA^t=I$
\\$det(AA^t)=det(A)det(A^t)=det(A)det(A)=det^2(A)=det(I)=1$. Portanto, $|det(A)|=1$.
}

\item[8] Sejam $\mathfrak{v}_1,...,\mathfrak{v}_m\text{ }\in \mathbb{R}^3$ linearmente independentes. Prove que \{$\mathfrak{v_1-v_2}$,
$\mathfrak{v_2-v_3}$,
...,$\mathfrak{v_{m-1}-v_{m}}$\} é L.I.

\item[9] Se $\mathfrak{v}_1,...,\mathfrak{v}_m\text{ gera } \mathbb{R}^n$, prove que \{$\mathfrak{v_1-v_2}$,
$\mathfrak{v_2-v_3}$,
...,$\mathfrak{v_{m-1}-v_{m}}$\} gera $\mathbb{R}^n$.

\item[10*] Sejam as matrizes $A_{m\times n}$, $B_{n\times p}$ e $C=AB$. Prove que $r(AB)\leq r(B)$.

\item[11] Mostre que nem toda matriz \textit{A} tal que $det(A)=1$ é ortogonal. Mostre que $
\left(\begin{array}{rr}
cos\theta & sen\theta \\
-sen\theta & cos\theta
\end{array}\right)
$ é ortogonal.
\\ \textcolor[rgb]{0,0,1}{
Como exemplo temos:
$
\left(\begin{array}{rr}
1 & 1 \\
1 & 2
\end{array}\right)
$
\\É ortogonal pois $<Cos\theta,-Sen\theta>\cdot<Sen\theta,Cos\theta>=0$
}

\item[12] Mostre que:
\begin{itemize}
\item[(a)] $
\left|\begin{array}{rrrr}
a&b&c&d\\
-1&x&0&0\\
0&-1&x&0\\
0&0&-1&x
\end{array}\right|=ax^3+bx^2+cx+d;
$
\textcolor[rgb]{0,0,1}{
$$
a\left|\begin{array}{rrr}
x&0&0\\
-1&x&0\\
0&-1&x
\end{array}\right|
-b\left|\begin{array}{rrr}
-1&0&0\\
0&x&0\\
0&-1&x
\end{array}\right|
+c\left|\begin{array}{rrr}
-1&x&0\\
0&-1&0\\
0&0&x
\end{array}\right|
-d\left|\begin{array}{rrr}
-1&x&0\\
0&-1&x\\
0&0&-1
\end{array}\right|
$$
\\$=ax^3+bx^2+cx+d$
\\
\item[(b)] $
\left|\begin{array}{rrrr}
1&-1&-1&-1\\
-1&1&-1&-1\\
-1&-1&1&-1\\
-1&-1&-1&1
\end{array}\right|=16.
$
$$
1\left|\begin{array}{rrr}
1&-1&-1\\
-1&1&-1\\
-1&-1&1
\end{array}\right|
-1\left|\begin{array}{rrr}
-1&-1&-1\\
-1&1&-1\\
-1&-1&1
\end{array}\right|
-1\left|\begin{array}{rrr}
-1&1&-1\\
-1&-1&-1\\
-1&-1&1
\end{array}\right|
-1\left|\begin{array}{rrr}
-1&1&-1\\
-1&-1&1\\
-1&-1&-1
\end{array}\right|
$$
\\$=1*(-4)-1*(-1)*(-4)-1*4-1*(-1)*(-4)=-16$
}
\end{itemize}
\end{itemize}

\section*{Lista 1: Matrizes e Sistemas Lineares}

\begin{itemize}
\item[1] Um exemplo de matriz $A \neq 0$ tal que $A^2=0$:
$$
\left[\begin{array}{rr}
a&b\\
c&d
\end{array}\right]
\quad
\left[\begin{array}{cc}
a&b\\
c&d
\end{array}\right]
=
\quad
\left[\begin{array}{cc}
0&0\\
0&0
\end{array}\right]
$$
\textcolor[rgb]{0,0,1}{Basta observar que se as linhas de uma matriz 2 por 2 possuem o mesmo valor em todos os elementos mas são de sinal trocado, temos uma matriz com quadrado nulo, conforme:
$$
k\left[\begin{array}{rr}
1&1\\
-1&-1
\end{array}\right]
\quad
k\left[\begin{array}{rr}
1&1\\
-1&-1
\end{array}\right]
=
k^2
\left[\begin{array}{cc}
0&0\\
0&0
\end{array}\right]
$$
Serve para qualquer valor de k diferente de zero!
}
\item[2] As matrizes quadradas de ordem n, A e B, comutam se $AB = BA$. 
\\\textcolor[rgb]{0,0,1}{As matrizes 2x2 que comutam com $\left[\begin{array}{cc}
1&0\\0&2
\end{array}\right]$ são:
$$
\begin{bmatrix}
1&0\\0&2
\end{bmatrix}
\begin{bmatrix}
a&b\\c&d
\end{bmatrix}
=
\begin{bmatrix}
a&b\\c&d
\end{bmatrix}
\begin{bmatrix}
1&0\\0&2
\end{bmatrix}
$$
\textbf{Com a = 0, c = 0, b = x e d = y onde $x,y\hspace{2mm}\in\hspace{2mm}\mathbb{R}^2$.}
\\As matrizes 3x3 que comutam com 
$\begin{bmatrix}
1&1&0\\1&1&1\\0&0&1
\end{bmatrix}$ são:
$$
\begin{bmatrix}
1&1&0\\1&1&1\\0&0&1
\end{bmatrix}
\begin{bmatrix}
a&b&c\\d&e&f\\g&h&i
\end{bmatrix}
=
\begin{bmatrix}
a&b&c\\d&e&f\\g&h&i
\end{bmatrix}
\begin{bmatrix}
1&1&0\\1&1&1\\0&0&1
\end{bmatrix}
$$
\textbf{Com b = f = a, i = e, d = 0, g = 0 e h = 0 onde $a,c,e\hspace{2mm}\in\hspace{2mm}\mathbb{R}^3$.}}

\item[3] Com auxílio do seguinte teorema:
\textcolor[rgb]{0,0,1}{\\ Se A é uma matriz quadrada e se existe um inteiro positivo k tal que $A^k=0$, então a matriz $I-A$ é invertível e $$(I-A)^{-1} = I + A + A^2 + ... + A^{k-1}$$
É possível aplicar facilmente para o caso em questão, onde $k=3$.
\begin{proof}
Por exemplo, se x é um número real qualquer e k é um inteiro positivo, então a identidade algébrica $$(1-x)(1+x+x^2+...+x^{k-1}$$ traduz para a seguinte identidade entre matrizes quadradas:
$$
(I-A)(I+A+A^2+...+A^{k-1})=I-A^k
$$
Isso pode ser confirmado multiplicando os fatores do lado esquerdo, pois temos o que se chada de soma telescópica (a origem de \textit{Telescoping series}):
$$
(I+A+A^2+...+A^{k-1}) - (A+A^2+A^3...+A^k) = I-A^k
$$
Com auxílio das equações acima pode-se concluir que no caso em questão, onde $k=3$ e $A^k=0$,      $I-A$ possui inversa (da definição) e seu valor é $I + A + A^2$ pois qualquer potência maior que três também vale zero quando $A^3=0$.
\end{proof}}

%%
%% itens não finalizados
%%

\item[4] Sejam A e B $n\times n$ tais que $A=A^t$, $B=B^t$, isto é, simétricas. Prove que AB é simétrica se, e só se, A e B comutam.
\textcolor[rgb]{0,0,1}{\\Se elas comutam, temos, em conjunto com a outra hipótese:
$$AB=BA=B^tA^t$$
Do teorema das transpostas\footnote{$(\mathbf{A B}^\mathrm{T}) = \mathbf{B}^\mathrm{T} \mathbf{A}^\mathrm{T}$}, e da equação acima, vemos que elas só são simétricas se comutam, conforme:
$$AB=B^tA^t=(AB)^t$$}

\item[5] Uma \textbf{matriz de probabilidade} é uma matriz $P=(p_{ij})_{n \times n}$ tal que $p_{ij} \geq 0$ pra todo par (i,j) e $\sum\limits_{j=1}^{n}(p_{ij})=1$ para todo \underline{i}. Prove que se \underline{P} e \underline{Q} são matrizes de probabilidade $n \times n$, então PQ também o é.

\textcolor[rgb]{0,0,1}{Se o produto PQ=C temos que, a soma de uma linha arbitrária é:
$$
\sum\limits_{l=1}^nc_{il}=\sum\limits_{l=1}^n\sum\limits_{k=1}^n p_{ik}q_{kl}=\sum\limits_{k=1}^n\sum\limits_{l=1}^n p_{ik}q_{kl}=\sum\limits_{k=1}^n p_{ik}\sum\limits_{l=1}^n q_{kl}
$$.
O termo constante durante o loop da soma em l pode ser retirado após a troca da prioridade no somatório. Observando a hipótese, onde a soma dos elementos de uma mesma linha é a unidade, temos:
$$
\sum\limits_{k=1}^n p_{ik}\sum\limits_{l=1}^n q_{kl}
=\sum\limits_{k=1}^n p_{ik}\times 1=1\times \sum\limits_{k=1}^n p_{ik} = 1 \times 1 = 1
$$
O produto de duas matrizes de probabilidade também possui a soma das linhas igual a unidade.
}
\item[6] Solução dos sistemas:
\textcolor[rgb]{0,0,1}{
\begin{itemize}
\item[a]
$ \left\{
\begin{array}{ll}
\displaystyle 2x_1 - 3x_2 + 7x_3 = 2 \\
\displaystyle 6x_1 - 9x_2 + 21x_3 = -1
\end{array}
\right.
 \left\{
\begin{array}{ll}
\displaystyle x_1 - \frac{3}{2}x_2 + \frac{7}{2}x_3 = 2 \\
\displaystyle \hspace{6mm} - 18x_2 \hspace{9mm} = -13
\end{array}
\right.
 \left\{
\begin{array}{ll}
\displaystyle x_1= \frac{37}{12} - \frac{7}{2}x_3 \\
\displaystyle x_2 = \frac{13}{18}
\end{array}
\right.
$
\item[b]
$ \left\{
\begin{array}{ll}
\displaystyle 6x_1+9x_2+14x_3=0 \\
\displaystyle 3x_1-8x_2+7x_3=0
\end{array}
\right.
 \left\{
\begin{array}{ll}
\displaystyle x_1+\frac{3}{2}x_2+\frac{7}{3}x_3=0 \\
\displaystyle \hspace{5mm}-\frac{25}{2}x_2 \hspace{9mm}=0
\end{array}
\right.
 \left\{
\begin{array}{ll}
\displaystyle x_1 =-\frac{7}{3}x_3 \\
\displaystyle x_2=0
\end{array}
\right.
$
\item[c]
$ \left\{
\begin{array}{llll}
\displaystyle x_1+x_2-x_3=-2 \\
\displaystyle 2x_1+x_2-3x_3=-7 \\
\displaystyle 4x_1-5x_2-13x_3=-35 \\
\displaystyle -3x_1+2x_2+8x_3=21
\end{array}
\right.
 \left\{
\begin{array}{llll}
\displaystyle x_1+x_2-x_3=-2 \\
\displaystyle \hspace{6mm} - x_2-x_3=-3 \\
\displaystyle \hspace{6mm} -9x_2-9x_3=-27 \\
\displaystyle \hspace{6mm}+5x_2+5x_3=15
\end{array}
\right.
 \left\{
\begin{array}{llll}
\displaystyle x_1+x_2-x_3=-2 \\
\displaystyle \hspace{6mm} x_2+x_3=3 \\
\displaystyle \hspace{6mm} x_2+x_3=3 \\
\displaystyle \hspace{6mm} x_2+x_3=3
\end{array}
\right.
$
\\
$
 \left\{
\begin{array}{ll}
\displaystyle x_1+x_2-x_3=-2 \\
\displaystyle \hspace{6mm} x_2+x_3=3
\end{array}
\right.
 \left\{
\begin{array}{ll}
\displaystyle x_1+x_2-x_3=-2 \\
\displaystyle \hspace{6mm} x_2=3-x_3
\end{array}
\right.
 \left\{
\begin{array}{ll}
\displaystyle x_1=2x_3 -5 \\
\displaystyle x_2=3-x_3
\end{array}
\right.
$
\item[d]
$ \left\{
\begin{array}{llll}
\displaystyle 2x_1+x_2-3x_3=0 \\
\displaystyle x_1-x_3=0 \\
\displaystyle -5x_1+x_2+2x_3=0 \\
\displaystyle 12x_1-x_2-x_3=0
\end{array}
\right.
 \left\{
\begin{array}{llll}
\displaystyle x_1-x_3=0 \\
\displaystyle x_2-x_3=0 \\
\displaystyle x_2-3x_3=0 \\
\displaystyle -x_2-11x_3=0
\end{array}
\right.
 \left\{
\begin{array}{llll}
\displaystyle x_1-x_3=0 \\
\displaystyle x_2-x_3=0 \\
\displaystyle -2x_3=0 \\
\displaystyle -12x_3=0
\end{array}
\right.
 \left\{
\begin{array}{lll}
\displaystyle x_1=0 \\
\displaystyle x_2=0 \\
\displaystyle x_3=0
\end{array}
\right.
$
\end{itemize}
}
\item[7] Calcule a e b de modo que sejam compatíveis com as equações abaixo:\\
\textcolor[rgb]{0,0,1}{
$ \left\{
\begin{array}{llll}
\displaystyle 4x_1+3x_2=-3 \\
\displaystyle 5x_1+4x_2=-5 \\
\displaystyle ax_1+2x_2=5 \\
\displaystyle 3x_1+bx_2=-1
\end{array}
\right.
 \left\{
\begin{array}{llll}
\displaystyle x_1+\frac{3}{4}x_2=-\frac{3}{4} \\
\displaystyle \hspace{8mm}\frac{1}{4}x_2=-\frac{5}{4} \\
\displaystyle \hspace{8mm}\frac{8-3a}{4}x_2=\frac{3a+20}{4} \\
\displaystyle \hspace{8mm}\frac{4b-9}{4}x_2=\frac{5}{4}
\end{array}
\right.
$
Observa-se que, para solução do sistema, $x_2=-5$. Assim:\\
$ \left\{
\begin{array}{ll}
\displaystyle -40+15a=3a+20 \\
\displaystyle -20b+45=5
\end{array}
\right.
 \left\{
\begin{array}{ll}
\displaystyle a=5 \\
\displaystyle b=2
\end{array}
\right.
$}

\item[8] Discuta como o posto de A varia com $t\in \mathbb{R}$, onde:
$$
A=\left[\begin{array}{ccc}
1&1&t\\1&t&1\\t&1&1
\end{array}\right]=
$$
\textcolor[rgb]{0,0,1}{
$$
\left[\begin{array}{ccc}
1&1&t\\0&t-1&1-t\\0&1-t&1-t^2
\end{array}\right]=
$$
Se $t=1$ o posto é igual a 1. Senão:
$$
=\left[\begin{array}{ccc}
1&1&t\\0&1&-1\\0&1&1+t
\end{array}\right]
=\left[\begin{array}{ccc}
1&1&t\\0&1&-1\\0&0&2+t
\end{array}\right]
$$
Se $t=-2$ o posto é 2. Caso contrário, o posto é 3.
}
\item[9] Discuta as soluções do sistema
$ \left\{
\begin{array}{ll}
\displaystyle x + ay = 1 \\
\displaystyle ax + y = a^3
\end{array}
\right.
$ conforme os valores do parâmetro $a \in \mathbb{R}.$
\\
\textcolor[rgb]{0,0,1}{Calculando a matriz escalonada reduzida, em função de a:
\\$
\left[\begin{array}{ccc}
1&a&1\\
0&1-a^2&a^3-a
\end{array}\right]
$
\\\\ Se $a=-1$:
\\$
\left[\begin{array}{ccc}
1&-1&1\\
0&0&0
\end{array}\right]
$, então: $<x,y>=<t+1,t>$ para qualquer $t\in \mathbb{R}$.
\\\\ Se $a=1$:
\\$
\left[\begin{array}{ccc}
1&1&1\\
0&0&0
\end{array}\right]
$, então: $<x,y>=<-t+1,t>$ para qualquer $t\in \mathbb{R}$.
\\\\ Nos outros casos:
\\$
\left[\begin{array}{ccc}
1&a&1\\
0&1&a
\end{array}\right]
$, então: $<x,y>=<1-a^2,a>$.
}

\item[10] Dê exemplo de sistema linear com 4 equações e 4 incógnitas cuja solução dependa de:
\begin{enumerate}
\item[a)] 1 parâmetro:
\\
\textcolor[rgb]{0,0,1}{$ \left\{
\begin{array}{lll}
\displaystyle x \hspace{12mm} - w = 0 \\
\displaystyle \hspace{6mm} y \hspace{6mm} - w = 0 \\
\displaystyle \hspace{12mm} z - w = 0
\end{array}
\right.
$}
\item[b)] 2 parâmetros
\\
\textcolor[rgb]{0,0,1}{$ \left\{
\begin{array}{ll}
\displaystyle x \hspace{12mm} - w = 0 \\
\displaystyle \hspace{6mm} y \hspace{6mm} - w = 0
\end{array}
\right.
$}
\item[c)] 3 parâmetros
\\
\textcolor[rgb]{0,0,1}{$ \left\{
\begin{array}{l}
\displaystyle x \hspace{12mm} - w = 0
\end{array}
\right.
$}
\end{enumerate}

\end{itemize}

%\section{Miscelânea: Exercícios Extras}
%
%\begin{theorem}
%\emph{Teorema 3.5.3: Equivalências entre solução do sistema homogêneo e não homogêneo.}
%
%\label{ANTON_3.5.3_equivalencias_sistemas}
%Se \textit{A} é uma matriz $m\times n$, então as seguintes afirmações são equivalentes.
%\begin{enumerate}
%\item $A\mathbf{x}=\mathbf{0}$ tem somente a solução trivial.
%\item $A\mathbf{x}=\mathbf{b}$ tem, no máximo, uma solução para cada \textbf{b} em $\mathbb{R}^m$ (ou seja, ou é inconsistente ou tem solução única).
%\end{enumerate}
%\end{theorem}
%
%\begin{proof}
%[ANTON-3.5.P1] Provando que $(a)\Rightarrow (b)$ no Teorema 3.5.3.
%\end{proof}
%
%\begin{proof} 
%[ANTON-3.5.P2] Provando que $(b)\Rightarrow (a)$ no Teorema 3.5.3. 
%\end{proof}
%
%[Sugestão seguida: Usar contraposição, supondo que o sistema não-homogêneo tem duas soluções distintas, e usando essas soluções para encontrar uma solução não trivial do sistema homogêneo.]
\end{document}

