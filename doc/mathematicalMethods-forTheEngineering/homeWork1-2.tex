\documentclass[11pt]{article}
\usepackage[utf8]{inputenc}
\usepackage{amsmath} %(?)
\usepackage{amsthm}%for theorem  
\usepackage{hyperref}
\usepackage{amssymb} % In LaTeX, how do I represent the hollow "R" symbol that designates the real number space? -> use \mathbb from amsfonts (loaded by amssymb)
%\title{Math by \LaTeX}
\usepackage{datetime}
%\newdateformat{mydate}{\monthname[\THEMONTH] \THEYEAR}
\newdateformat{mydate}{\THEDAY /\THEMONTH /\THEYEAR}

\title{Exercicios da Primeira Etapa - Prova I}
\date{\mydate\today}
\author{Luiz Le Roy}
\begin{document}
  \maketitle

\section*{Espaços Vetoriais Arbitrários}
\begin{itemize}
\item[1] O conjunto das funções $f:\mathbb{R}\longrightarrow \mathbb{R}$ duas vezes continuamente deriváveis e tais que $f''+af'+bf=0$, onde $\underline{a}$ e $\underline{b}$ são reais fixos, munido das leis usuais, é um espaço vetorial?

\item[2] Prove que o conjunto das funções limitadas $f:\mathbb{R}\longrightarrow \mathbb{R}$ (isto é, tais que existe $M_f>0$ tal que $|f(x)|\leq M_f$ para todo $x\in \mathbb{R}$), com as leis usuais, é um espaço vetorial.

\item[3] Seja $V=\mathfrak{F}(\mathbb{R,R})$. Verifique se W é subespaço de V:
\begin{itemize}
\item [a)] W é $f(-x)=f(x)$
\item [b)] W é $f(-x)=-f(x)$
\item[c)] W é o conjunto das funções deriváveis.
\end{itemize}

\item[4] Mostre que $W=\{(0,y,z)\in \mathbb{R}^3;y,z \in \mathbb{R}\}$ é subespaço  de $\mathbb{R}^3$ gerado por (0,1,1) e (0,2,-1).

\item[5] Mostre que $p(t)=t^3-t^2+1$, $q(t)=t^2-1$ e $r(t)=2t^3+t-1$ são L.I. em $P_4$.

\item[6] Prove que $f(t)=t$, $g(t)=e^t$ e $h(t)=sen(t)$ são L.I.

\item[7] Seja W o subespaço de $P_4$ gerado por $u=t^3-t^2+1$, $v=t^2-1$ e $w=t^3-3t^2+3$. Ache uma base de W.

\item[8] Existe base de $P_4$ que não contenha nenhum polinômio de grau 2?

\item[9] Seja $dimV=n$. Prove que se $x={v_1,v_2,...,v^p}$ gera V, então X contém uma base de V.

\item[10] Prove que se $v_1,...,v_n$ geram V e $p>n$, então $u_1,...,u_p\in V$ são L.D.

\item[11] Dado o conjunto finito $X=\{a_1,a_2,...,a_n\}$, ache uma base para o espaço vetorial $V=\mathfrak{F}(X,\mathbb{R})=\{f:X\longrightarrow \mathbb{R}$.

\item[12] Ache $T:\mathbb{R}^2\longrightarrow \mathbb{R}$ tal que $T(1,1)=-1$ e $T(1,0)=3$, T linear.

\item[13] Ache $T:\mathbb{R}^3\longrightarrow \mathbb{R}^4$ linear tal que ImT seja gerada por (1,0,2,-4) e (0,2,-1,3).

\item[14] Seja $T:V\longrightarrow V$ linear. Prove que se $T(v_1),...,T(v_n)$ são L.I., então $v_1,...,v_n$ são L.I.

\item[15] Sejam L, $T:V\longrightarrow V$ isomorfismos. Prove: $(L\circ T)^{-1}=T^{-1}\circ L^{-1}$.
\end{itemize}

\section*{L.I e L.D. Determinantes e Matrizes}

\begin{itemize}
\item[1] Sejam $A_{m \times n}$, $P_{m \times m}$ invertível, $\textbf{a}_1, \textbf{a}_2, ..., \textbf{a}_r$ r colunas L.I de A. Prove que $P\textbf{a}_1,P\textbf{a}_2,...,P\textbf{a}_r$ são r colunas L.I. de PA.
\begin{proof}
É preciso da seguinte equação:
\[
a^2 + b^2 = c^2
\]
\textit{Thats all folks!} \qedhere
\end{proof}

\item[2] Discuta as soluções do sistema:
\begin{eqnarray*} 
ax \hspace{9mm} +bz &=&1\\
ax+ay+4z&=&2\\
ay+2z&=&3 
\end{eqnarray*}

Com auxílio da matriz aumentada, e operações como permutação e combinações de linhas temos a seguinte sequência, válida apenas para $a\neq0$ e $b\neq2$:

$$
\begin{bmatrix}
a&0&b&1\\0&a&2&3\\a&a&4&2
\end{bmatrix}
\begin{bmatrix}
1&0&\frac{b}{a}&\frac{1}{a}\\0&1&\frac{2}{a}&\frac{3}{a}\\0&a&4-b&1
\end{bmatrix}
\begin{bmatrix}
1&0&\frac{b}{a}&\frac{1}{a}\\0&1&\frac{2}{a}&\frac{3}{a}\\0&0&2-b&-2
\end{bmatrix}
\Rightarrow
$$
$$
\Rightarrow
\begin{bmatrix}
x\\y\\z
\end{bmatrix}
=
\begin{bmatrix}
\frac{1}{a}-\frac{3b}{a^2} + \frac{4b}{a^2(b-2)}\\\frac{3}{a}-\frac{4}{a(b-2)}\\\frac{2}{b-2}
\end{bmatrix}
$$
Se $a=0$ ou $b=2$ o sistema é impossível.

\item[3] Sem calcular o determinante, prove que 
$ \left \vert
\begin{array}{lll}
\displaystyle \text{1 2 3} \\
\displaystyle \text{4 5 6} \\
\displaystyle \text{7 8 9}
\end{array}
\right \vert = 0.
$

\item[4] Prove que
$ v(a,b,c) = \left \vert
\begin{array}{lll}
\displaystyle 1&1&1 \\
\displaystyle a&b&c \\
\displaystyle a^2&b^2&c^2
\end{array}
\right \vert = (c-a)(c-b)(b-a).
$

\item[5] Seja $A_{m\times n}$. Se $P_{n\times n}$ é invertível e $B=P^-1AP$, prove que $det(A)=det(B)$, e que $det(B-\lambda I_n)=det(A-\lambda I_n)$ qualquer que seja $\lambda \in \mathbb{R}$.

\item[6] Sejam $\mathfrak{v}_1,...,\mathfrak{v}_n\text{ }\in \mathbb{R}^n$, $\mathfrak{v}_j=\sum_{i=1}^n a_{ij}e_i$, $A=(a_{ij})$, $\mathcal{E}=(e_1,...,e_n)$ a base canônica do $\mathbb{R}^n$. Prove que $\mathfrak{v}_1,...,\mathfrak{v}_n$ são L.I. se, e só se, $det(A)\neq0$.

\item[7] Uma matriz quadrada H é \underline{ortogonal} se $HH^t=I_n$. Prove que o produto de duas matrizes ortogonais é uma matriz ortogonal, e que $det(H)=\pm 1$.

\item[8] Sejam $\mathfrak{v}_1,...,\mathfrak{v}_m\text{ }\in \mathbb{R}^3$ linearmente independentes. Prove que \{$\mathfrak{v_1-v_2}$,
$\mathfrak{v_2-v_3}$,
...,$\mathfrak{v_{m-1}-v_{m}}$\} é L.I.

\item[9] Se $\mathfrak{v}_1,...,\mathfrak{v}_m\text{ gera } \mathbb{R}^n$, prove que \{$\mathfrak{v_1-v_2}$,
$\mathfrak{v_2-v_3}$,
...,$\mathfrak{v_{m-1}-v_{m}}$\} gera $\mathbb{R}^n$.

\item[10] Sejam as matrizes $A_{m\times n}$, $B_{n\times p}$ e $C=AB$. Prove que $r(AB)\leq r(B)$.

\item[11] Mostre que nem toda matriz \textit{A} tal que $det(A)=1$ é ortogonal. Mostre que $
\left(\begin{array}{rr}
cos\theta & sen\theta \\
-sen\theta & cos\theta
\end{array}\right)
$ é ortogonal.

\item[12] Mostre que:
\begin{itemize}
\item[(a)] $
\left|\begin{array}{rrrr}
a&b&c&d\\
-1&x&0&0\\
0&-1&x&0\\
0&0&-1&x
\end{array}\right|=ax^3+bx^2+cx+d;
$
\item[(b)] $
\left|\begin{array}{rrrr}
1&-1&-1&-1\\
-1&1&-1&-1\\
-1&-1&1&-1\\
-1&-1&-1&1
\end{array}\right|=16.
$
\end{itemize}
\end{itemize}

\section*{Matrizes e Sistemas Lineares}

\begin{itemize}
\item[1] Um exemplo de matriz $A \neq 0$ tal que $A^2=0$:
$$
\left[\begin{array}{rr}
a&b\\
c&d
\end{array}\right]
\quad
\left[\begin{array}{cc}
a&b\\
c&d
\end{array}\right]
=
\quad
\left[\begin{array}{cc}
0&0\\
0&0
\end{array}\right]
$$
\textbf{Onde a = 1, b = 1, c = -1 e d = -1.}
\item[2] As matrizes quadradas de ordem n, A e B, comutam se $AB = BA$. 
\\As matrizes 2x2 que comutam com $\left[\begin{array}{cc}
1&0\\0&2
\end{array}\right]$ são:
$$
\begin{bmatrix}
1&0\\0&2
\end{bmatrix}
\begin{bmatrix}
a&b\\c&d
\end{bmatrix}
=
\begin{bmatrix}
a&b\\c&d
\end{bmatrix}
\begin{bmatrix}
1&0\\0&2
\end{bmatrix}
$$
\textbf{Com a = 0, c = 0, b = x e d = y onde $x,y\hspace{2mm}\in\hspace{2mm}\mathbb{R}^2$.}
\\As matrizes 3x3 que comutam com 
$\begin{bmatrix}
1&1&0\\1&1&1\\0&0&1
\end{bmatrix}$ são:
$$
\begin{bmatrix}
1&1&0\\1&1&1\\0&0&1
\end{bmatrix}
\begin{bmatrix}
a&b&c\\d&e&f\\g&h&i
\end{bmatrix}
=
\begin{bmatrix}
a&b&c\\d&e&f\\g&h&i
\end{bmatrix}
\begin{bmatrix}
1&1&0\\1&1&1\\0&0&1
\end{bmatrix}
$$
\textbf{Com b = f = a, i = e, d = 0, g = 0 e h = 0 onde $a,c,e\hspace{2mm}\in\hspace{2mm}\mathbb{R}^3$.}

\item[3] Com auxílio do seguinte teorema:
\\ Se A é uma matriz quadrada e se existe um inteiro positivo k tal que $A^k=0$, então a matriz $I-A$ é invertível e $$(I-A)^{-1} = I + A + A^2 + ... + A^{k-1}$$
É possível aplicar facilmente para o caso em questão, onde $k=3$.
\begin{proof}
Por exemplo, se x é um número real qualquer e k é um inteiro positivo, então a identidade algébrica $$(1-x)(1+x+x^2+...+x^{k-1}$$ traduz para a seguinte identidade entre matrizes quadradas:
$$
(I-A)(I+A+A^2+...+A^{k-1})=I-A^k
$$
Isso pode ser confirmado multiplicando os fatores do lado esquerdo, pois temos o que se chada de soma telescópica (a origem de \textit{Telescoping series}):
$$
(I+A+A^2+...+A^{k-1}) - (A+A^2+A^3...+A^k) = I-A^k
$$
Com auxílio das equações acima pode-se concluir que no caso em questão, onde $k=3$ e $A^k=0$,      $I-A$ possui inversa (da definição) e seu valor é $I + A + A^2$ pois qualquer potência maior que três também vale zero quando $A^3=0$.\qedhere
\end{proof}

%%
%% itens não finalizados
%%

\item[4] Sejam A e B $n\times n$ tais que $A=A^t$, $B=B^t$, isto é, simétricas. Prove que AB é simétrica se, e só se, A e B comutam.

Se elas comutam, temos, em conjunto com a outra hipótese:
$$AB=BA=B^tA^t$$
Do teorema das transpostas\footnote{$(\mathbf{A B}^\mathrm{T}) = \mathbf{B}^\mathrm{T} \mathbf{A}^\mathrm{T}$}, e da equação acima, vemos que elas só são simétricas se comutam, conforme:
$$AB=B^tA^t=(AB)^t$$

\item[5] Uma \textbf{matriz de probabilidade} é uma matriz $P=(p_{ij})_{n \times n}$ tal que $p_{ij} \geq 0$ pra todo par (i,j) e $\sum_{j=1}^{n}(p_{ij})=1$ para todo \underline{i}. Prove que se \underline{P} e \underline{Q} são matrizes de probabilidade $n \times n$, então PQ também o é.

\item[6] Resolva os sistemas:
\begin{itemize}
\item[a]
$ \left\{
\begin{array}{ll}
\displaystyle 2x_1 - 3x_2 + 7x_3 = 2 \\
\displaystyle 6x_1 - 9x_2 + 21x_3 = -1
\end{array}
\right.
$
\item[b]
$ \left\{
\begin{array}{ll}
\displaystyle 6x_1+9x_2+14x_3=0 \\
\displaystyle 3x_1-8x_2+7x_3=0
\end{array}
\right.
$
\item[c]
$ \left\{
\begin{array}{llll}
\displaystyle x_1+x_2-x_3=-2 \\
\displaystyle 2x_1+x_2-3x_3=-7 \\
\displaystyle 4x_1-5x_2-13x_3=-35 \\
\displaystyle -3x_1+2x_2+8x_3=21
\end{array}
\right.
$
\item[d]
$ \left\{
\begin{array}{llll}
\displaystyle 2x_1+x_2-3x_3=0 \\
\displaystyle x_1-x_3=0 \\
\displaystyle -5x_1+x_2+2x_3=0 \\
\displaystyle 12x_1-x_2-x_3=0
\end{array}
\right.
$
\end{itemize}

\item[7] Calcule a e b de modo que sejam compatíveis com as equações:\\
$ \left\{
\begin{array}{llll}
\displaystyle 4x_1+3x_2=-3 \\
\displaystyle 5x_1+4x_2=-5 \\
\displaystyle ax_1+2x_2=5 \\
\displaystyle 3x_1+bx_2=-1
\end{array}
\right.
$

\item[8] Discuta como o posto de A varia com teR, onde:

\item[9] Discuta as soluções do sistema
$ \left\{
\begin{array}{ll}
\displaystyle x - ay = 1 \\
\displaystyle ax + y = a^3
\end{array}
\right.
$ conforme os valores do parâmetro $a \in \mathbb{R}.$

\item[10] Dê exemplo de sistema linear com 4 equações e 4 incógnitas cuja solução dependa de:
\begin{enumerate}
\item[a)] 1 parâmetro
\item[b)] 2 parâmetros
\item[c)] 3 parâmetros
\end{enumerate}

\end{itemize}
								
\end{document}

