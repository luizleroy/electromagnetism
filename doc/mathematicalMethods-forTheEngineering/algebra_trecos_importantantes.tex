\documentclass[a4paper]{article}

\usepackage[english]{babel}
\usepackage[utf8]{inputenc}
\usepackage{amsmath}
\usepackage{graphicx}
\usepackage[colorinlistoftodos]{todonotes}
\usepackage{enumitem}% http://ctan.org/pkg/enumitem

\newtheorem{theorem}{Teorema}[section]
\newtheorem{lemma}[theorem]{Lema}
\newtheorem{proposition}[theorem]{Proposition}
\newtheorem{corollary}[theorem]{Corollary}
\newenvironment{proof}[1][Prova]{\begin{trivlist}
\item[\hskip \labelsep {\bfseries #1}]}{\end{trivlist}}
\newenvironment{definition}[1][Definição]{\begin{trivlist}
\item[\hskip \labelsep {\bfseries #1}]}{\end{trivlist}}
\newenvironment{example}[1][Example]{\begin{trivlist}
\item[\hskip \labelsep {\bfseries #1}]}{\end{trivlist}}
\newenvironment{remark}[1][Remark]{\begin{trivlist}
\item[\hskip \labelsep {\bfseries #1}]}{\end{trivlist}}
\newcommand{\qed}{\nobreak \ifvmode \relax \else
      \ifdim\lastskip<1.5em \hskip-\lastskip
      \hskip1.5em plus0em minus0.5em \fi \nobreak
      \vrule height0.75em width0.5em depth0.25em\fi}

\title{Métodos Matemáticos\\O que há de importante na Álgebra Abstrata}
\author{Luiz Le Roy}
\date{\today}

\begin{document}
\maketitle

%\begin{abstract}
%Your abstract.
%\end{abstract}

\section{Definições e Exemplos}
Uma importantíssima definição descreve o que é um espaço vetorial, como se segue \cite{anton}:
\begin{definition}
Seja \textit{V} um conjunto não vazio de objetos sobre os quais estão definidas operações de adição vetorial e multiplicação por escalar. Por \textbf{adição vetorial} queremos dizer uma regra que associa a cada par de objetos \textbf{u} e \textbf{v} de \textit{V} um único objeto $\textbf{u}+\textbf{v}$ que consideramos como a soma de \textbf{u} com \textbf{v} e por \textbf{multiplicação por escalar} queremos dizer uma regra que associa a cada escalar \textit{a} e a cada objeto \textbf{u} de \textit{V} um único objeto $\textit{a}\textbf{u}$ que consideramos como a multiplicação de \textbf{u} pelo escalar \textit{a}. O conjunto \textit{V} munido dessas operações será denominado um \textbf{espaço vetorial} e os objetos de \textit{V} serão denominados \textbf{vetores} se valeram as seguintes propriedades para quaisquer \textbf{u}, \textbf{v} e \textbf{w} de \textit{V} e quaisquer escalares \textit{a} e \textit{b}.
\begin{enumerate}
\item V é \textbf{fechado na adição}, ou seja, se \textbf{u} e \textbf{v} estão em \textit{V}, então $\textbf{u}+\textbf{v}$ está em \textit{V}.
\item $ \mathbf{u}+\mathbf{v}=\mathbf{v}+\mathbf{u}$
\item $ (\mathbf{u}+\mathbf{v})+\mathbf{w}=\mathbf{u}+(\mathbf{v}+\mathbf{w})$
\item \textit{V} contém um objeto \textbf{0} (que denominamos \textbf{vetor nulo} ou \textbf{vetor zero}) que se comporta como um zero aditivo no seguinte sentido: $\mathbf{u}+\mathbf{0}=\mathbf{u}$ para cada $\mathbf{u}$ em $\mathit{V}$.
\item Para cada objeto \textbf{u} em \textit{V} existe um objeto $-\mathbf{u}$ em \textit{V} (que denominamos um \textbf{negativo} de \textbf{u}) tal que $\mathbf{u}+\mathbf{(-u)}=\mathbf{0}$.
\item \textit{V} é \textbf{fechado na multiplicação por escalar}, ou seja, se \textbf{u} está em \textit{V} e \textit{a} é um escalar, então $\textit{a}\textbf{u}$ está em \textit{V}.
\item $\mathit{a}(\mathbf{u+v})=\mathit{a}\mathbf{u}+\mathit{a}\mathbf{v}$
\item $(\mathit{a+b})\mathbf{u}=\mathit{a}\mathbf{u}+\mathit{b}\mathbf{u}$
\item $\mathit{a}(\mathit{b}\mathbf{u})=(\mathit{a}\mathit{b})\mathbf{u}$
\item $1\mathbf{u}=\mathbf{u}$
\end{enumerate}
\end{definition}
\begin{definition}
Uma função $T:R^n \longrightarrow R^m$ é dita uma \textbf{transformação linear} de $R^n$ em $R^m$ se as propriedades seguintes valem para quaisquer vetores \textbf{v} e \textbf{w} de $R^n$ e qualquer escalar \textit{c}:
\begin{enumerate}[label=\roman*]
\item $T(c\textbf{v})=cT(\textbf{v})$ \textbf{Homogeneidade}
\item $T(\textbf{v}+\textbf{w})= T(\textbf{v}) + T(\textbf{w})$ \textbf{Aditividade}
\end{enumerate}
No caso especial em que m=n, a transformação linear T é denominada um \textbf{operador linear} de $R^n$.
\end{definition}

\begin{theorem}
\label{Desigualdade_Quadrados_Inversos}
$$
(a^2_1+a^2_2+a^2_3+ ... +a^2_n) \ast \left(\frac{1}{a^2_1}+\frac{1}{a^2_2}+\frac{1}{a^2_3}+ ... + \frac{1}{a^2_n}\right) \geq n^2
$$
\end{theorem}
\bibliographystyle{abnt}
\bibliography{mmb}
\end{document}
