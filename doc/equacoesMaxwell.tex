\documentclass [a4paper,12pt]{article}
\usepackage [brazil]{babel}
\usepackage [utf8]{inputenc}

\begin{document}

\section{Equações de Maxwell}
Pode-se pensar em reescrever as equações de Maxwell de uma nova maneira, sem anecessidade de uma `métrica espacial' intrínseca:

Equação de Faraday:
$$
\partial E + \frac{\partial{B}}{\partial{t}} = 0
$$

Lei de Ampère-Maxwell:
$$
\partial H = \frac{\partial{D}}{\partial{t}} + J
$$

Lei de Gaus:
$$
\partial{D} = \rho
$$

Lei de Gaus do Magnetismo:
$$
\partial{B} = 0
$$

A métrica aparecerá nas relações constitutivas:
$$
B=\mu H
$$
$$
D = \epsilon E
$$
$$
J_c=\sigma E
$$



Abaixo encontra-se as equações no formato dif-curl (para comparação):
$$
\nabla\times\textbf E + \frac{\partial\textbf B}{\partial t} = 0
$$
$$
\nabla\times\textbf H = \frac{\partial\textbf D}{\partial t} + \textbf J
$$
$$
\nabla\cdot\textbf D = \rho
$$
$$
\nabla\cdot\textbf B = 0
$$

\end{document}
