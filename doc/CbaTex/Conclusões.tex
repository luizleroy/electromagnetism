\section{Conclus�es} \label{con}

%Se o limite de perda ativa m�nima n�o � observado a instala��o de mais bancos de 
%capacitores, apesar de elevar ainda mais o perfil de tens�o, proporciona inje��o de pot�ncia 
%reativa capacitiva que faz aumentar o valor da corrente total do sistema e, portanto, aumentar o 
%valor das perda

Observa-se que, ao inserir grupos de capacitores em pontos espec�ficos da rede,
� poss�vel reduzir perdas ativas e reativas no sistema diminuindo a corrente no tronco
principal do alimentador. No in�cio de circuito, antes da contabiliza��o de cargas, a corrente, em m�dulo, foi reduzida de x,xkMM para y,ykMM. Em paralelo, a varia��o no n�vel de tens�o
ao longo das barras fica est�vel, trazendo os valores para a proximidade da
tens�o nominal (1,0pu), conforme visto na se��o de resultados. O fator de destaque no trabalho foi a utiliza��o de capacitores com o objetivo principal de minimizar perdas ao longo do circuito. Os benef�cios secund�rios, na corrente el�trica ao longo do circuito e a estabiliza��o do perfil de tens�o, foram importantes para justificar os gastos com a inclus�o de novos equipamentos de pot�ncia na linha.
