\section{Conclus�es} \label{con}
Durante este trabalho, o objetivo foi minimizar perdas el�tricas do circuito de distribui��o de energia e estabilizar o n�vel de tens�o ao longo das barras do alimentador. Optou-se por utilizar uma minimiza��o evolucion�ria para encontrar, separadamente, pontos �timos de projeto na rede que atenda aos dois crit�rios.

Destaca-se no trabalho a solu��o do problema de aloca��o de capacitores em redes de distribui��o extensas. Esta abordagem permite a automatiza��o do processo e demanda menor interfer�ncia de um especialista na an�lise do problema. As demandas de n�vel de tens�o descritas em normas foram atingidas para um circuito radial de 2498 barras. As perdas el�tricas, neste mesmo circuito, foram para patamares aceit�veis ap�s a inser��o de 5 bancos de capacitores trif�sicos de 150 kVAr.

A rede foi carrega com capacitores totalizando 750 kVAr e n�o apresentou nenhuma viola��o, sendo que, inicialmente, existiam 231 viola��es de tens�o no limite superior e 173 no limite inferior, por fase.

Este tipo de abordagem capacita a concession�ria � distribuir melhor os recursos quando o n�mero de barras e linhas do circuito � superior a uma centena de unidades. Nestes casos uma an�lise t�cnica ponto a ponto � praticamente invi�vel, portanto processos automatizados s�o muito vantajosos.