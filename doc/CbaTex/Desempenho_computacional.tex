\section{Desempenho computacional} \label{comput}
Todos os c�lculos para o fluxo de carga, e a otimiza��o por m�todos evolucion�rios, foram codificados em C ANSI, devido a necessidade de desempenho. Julga-se que, neste tipo de trabalho, onde os par�metros de entrada variam muito a performance, diversos testes em situa��es extremas s�o necess�rios.

Portanto, fun��es de teste com os valores �timos conhecidos, como o vale de Rosenbrock, a fun��o de De Jong, fun��o de Rastrigin, fun��o de Schwefel entre outras foram utilizadas para calibrar o algoritmo desenvolvido. Nesta etapa, os par�metros de opera��o do PSO, descritos na se��o \ref{rev}, foram encontrados. Utilizou-se \citeasnoun{molga2005test} para esta an�lise.

 O hardware dispon�vel para as simula��es deste trabalho, possui um processador da Intel\textregistered, com o E5-2630 xeon\textregistered, com o clock de 2.30GHz, capaz de executar 12 threads. No PSO, pode-se realizar simultaneamente uma grande quantidade de c�lculos em paralelo. Portanto uma estrat�gia de paralelismo foi constru�da. 

As fun��es objetivo de cada uma das gera��es do algoritmo foram calculadas com paralelismo, com aux�lio da API OpenMP\textregistered, descrita em \citeasnoun{chapman2008using}. A escolha desta biblioteca se deve ao hardware utilizado, que � de mem�ria compartilhada. Os resultados de ganho de desempenho, em termos absolutos, e o \textit{speedup}\footnote{Em computa��o, \textit{speedup} se refere ao quanto um algoritmo � mais r�pido que sua vers�o sequencial.} est�o resumidos na \Tab \ref{tab_desempenho}.

%In parallel computing, speedup refers to how much a parallel algorithm is faster than a corresponding sequential algorithm. RODAP�?
\begin{table}[ht!]
  \centering
  \begin{tabular}{| l | l | l |}
  \hline
  Dura��o do processo & Threads & Speedup\\
  \hline
  10.885,378s $\approx$ 3h & 1 & 1 \\
  02.613,357s $\approx$ 43min & 8 & 4,165 \\
  01.105,509s $\approx$ 18min & 12 & 9,846 \\
  \hline
\end{tabular}
  \caption{Teste de desempenho}
  \label{tab_desempenho}
\end{table}

% & Intel Core i5 3.10GHz
% & Intel Core i7 2 GHz
% & Intel Xeon e5 2.30GHz