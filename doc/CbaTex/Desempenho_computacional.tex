\section{Desempenho computacional} \label{comput}
Todos os c�lculos para o fluxo de carga, e a otimiza��o por m�todos evolucion�rios, foram codificados em C ANSI. Um dos focos deste trabalho foi o desempenho. Diversos testes em situa��es extremas (no que tange a performance) foram realizados.

Fun��es de teste com os valores �timos conhecidos, como o vale de Rosenbrock, a fun��o de De Jong, fun��o de Rastrigin, fun��o de Schwefel entre outras foram utilizadas para calibrar o algoritmo desenvolvido. Nesta etapa, os par�metros de opera��o do PSO, descritos na se��o \ref{rev}, foram encontrados. Utilizou-se \citeasnoun{molga2005test} para esta an�lise.

Julga-se que, neste tipo de trabalho, onde os par�metros de entrada variam muito a performance, um teste de desempenho � necess�rio. O hardware dispon�vel para as simula��es deste trabalho, foi um servidor de 24 n�cleos da Intel\textregistered, com o E5-2630 xeon\textregistered, de 2.30GHz cada. Que favorece o paralelismo de n�cleo. Somando ao fato de que, no PSO, pode-se realizar simultaneamente uma grande quantidade de c�lculos, uma estrat�gia de paralelismo foi constru�da. 

Todas as fun��es objetivo de uma gera��o foram calculadas em paralelo, com aux�lio da API OpenMP\textregistered, descrita em \citeasnoun{chapman2008using}. A escolha desta biblioteca se deve ao hardware utilizado, que � de mem�ria compartilhada. Os resultados de ganho de desempenho, em termos absolutos, est�o resumidos na \Tab \ref{tab_desempenho}.

%In parallel computing, speedup refers to how much a parallel algorithm is faster than a corresponding sequential algorithm. RODAP�?
\begin{table}[ht!]
  \centering
  \begin{tabular}{| l | l | l |}
  \hline
  Dura��o do processo & N�cleos & $\frac{serial}{paralelo}$\\
  \hline
  10.885,378s $\approx$ 3h & serial & - \\
  02.613,357s $\approx$ 43min & 8 & 4,165 \\
  01.105,509s $\approx$ 18min & 24 & 9,846 \\
  \hline
\end{tabular}
  \caption{Teste de desempenho}
  \label{tab_desempenho}
\end{table}

% & Intel Core i5 3.10GHz
% & Intel Core i7 2 GHz
% & Intel Xeon e5 2.30GHz