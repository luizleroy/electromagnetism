%\documentclass{article}
%\usepackage{algorithm,algorithmic}
%\usepackage{caption}
%
%\newlength\myindent
%\setlength\myindent{2em}
%\newcommand\bindent{%
%  \begingroup
%  \setlength{\itemindent}{\myindent}
%  \addtolength{\algorithmicindent}{\myindent}
%}
%\newcommand\eindent{\endgroup}
%
%\begin{document}
%
%\begin{algorithm}[H]
%\caption*{my algorithm}
%\begin{algorithmic} 
%\REQUIRE load flow
%\ENSURE optimal OCP
%\STATE[1] Calcular constante de inércia com Eq. x
%\STATE[2] Para todas as partículas:
%	\STATE[3]Calcular velocidade com Eq. x
%\end{algorithmic}
%\end{algorithm}
%
%\end{document}

\documentclass{article}
\usepackage[latin1]{inputenc}
\usepackage{amsmath}
\usepackage{hyperref}

\usepackage{algorithm}
\usepackage{algorithmic}

\title{\LaTeX}
\date{}
\author{Luiz Le Roy}

%descobirir como trocar o nome do ambiente...
%\newcommand{\INDSTATE}[1][1]{\STATE\hspace{#1\algorithmicindent}}
%\newenvironment{myalgorithm}[1][htb]
%  {\renewcommand{\algorithmcfname}{Algoritmo}%
%   \begin{algorithm}[#1]
%  }{\end{algorithm}}

\begin{document}
  \maketitle
  
\begin{algorithm}
\caption{Algoritmo OCP via PSO}
\begin{algorithmic} 
\REQUIRE load flow
\ENSURE optimal OCP
\STATE[1] Calcular constante de inércia com Eq. x
\STATE[2] Para todas as partículas:
	\STATE[3]Calcular velocidade com Eq. x
\end{algorithmic}
\end{algorithm}
								
\end{document}