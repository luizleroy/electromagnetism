\documentclass{IEEEtran}
\usepackage{filecontents}
\usepackage{lipsum}
\usepackage[utf8]{inputenc}
\usepackage{graphicx} % for image
\usepackage{color} % for image (?)
\usepackage{amsmath}
\DeclareGraphicsExtensions{.png}

% correct bad hyphenation here
\hyphenation{op-tical net-works semi-conduc-tor}


\begin{document}

\title{SSA\\Otimizações em Linhas de Distribuição de Energia via um Algoritmo Evolucionário Denominado \textit{Particle Swarm Optimization}\\\textbf{Bibliografia Preliminar}}
%
%
% author names and IEEE memberships
% note positions of commas and nonbreaking spaces ( ~ ) LaTeX will not break
% a structure at a ~ so this keeps an author's name from being broken across
% two lines.
% use \thanks{} to gain access to the first footnote area
% a separate \thanks must be used for each paragraph as LaTeX2e's \thanks
% was not built to handle multiple paragraphs
%

\author{Luiz~Le~Roy - ~\IEEEmembership{PUC Minas}
        % <-this % stops a space
\thanks{Texto iniciado em 4 de abril de 2014.}}

% The paper headers
\markboth{Journal of \LaTeX\,~Vol.~X, No.~Y, Abril~2014}%
{Shell \MakeLowercase{\textit{et al.}}: Bare Demo of IEEEtran.cls for Journals}

% make the title area
\maketitle

\begin{abstract}
Levantamento da bibliografia.
\end{abstract}

\begin{IEEEkeywords}
PSO \textit{(Particle Swarm Optimization)}.
\end{IEEEkeywords}

\IEEEpeerreviewmaketitle

\section{Introdução}
\IEEEPARstart{A}{pós} conversar com o orientador fizemos o seguinte combinado:
Devemos concentrar em duas diretivas:
\begin{itemize}[]
\item Buscar um problema de maior ganho para a Cemig, e
\item encontrar algo que pode trazer ganhos para concessionárias de energia de forma geral.
\end{itemize}

No primeiro caso, podemos tentar encontrar uma contribuição dentro de uma realidade nacional, que possui diversas peculiaridades. No segundo item, podemos abordar mais as inúmeras possibilidades de contribuições científicas.

\section{Referência para primeira apresentação}
Seguindo a ordem natural, explorei \cite{del2008particle} para familiarizar com os problemas de SEP, especificamente a Distribuição de Energia, com auxílio da ferramenta de computação evolucionária PSO - \textit{Particle Swarm Optimization}.



O que produzi até agora foi: \cite{arruda2005calculation}, \cite{adriano2006modelos} e \cite{adriano2006efeitos}.


\section{Conclusão}
Até o momento, foi possível concluir coisas interessantes sobre a alocação estática de dispositivos capacitivos em linhas de distribuição?



% Can use something like this to put references on a page
% by themselves when using endfloat and the captionsoff option.
\ifCLASSOPTIONcaptionsoff
  \newpage
\fi

\bibliographystyle{ieeetran}
\bibliography{bibliography}

\begin{IEEEbiographynophoto}{Luiz~Le~Roy}
é estudante do curso de mestrado da Pontif\'icia Universidade Cat\'olica de Minas Gerais.
\end{IEEEbiographynophoto}

% that's all folks
\end{document}
