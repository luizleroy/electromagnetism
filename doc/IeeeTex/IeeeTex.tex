%% bare_jrnl.tex
%% V1.3
%% 2007/01/11
%% by Michael Shell
%% see http://www.michaelshell.org/
%% for current contact information.
%%
%% This is a skeleton file demonstrating the use of IEEEtran.cls
%% (requires IEEEtran.cls version 1.7 or later) with an IEEE journal paper.
%%
%% Support sites:
%% http://www.michaelshell.org/tex/ieeetran/
%% http://www.ctan.org/tex-archive/macros/latex/contrib/IEEEtran/
%% and
%% http://www.ieee.org/



% *** Authors should verify (and, if needed, correct) their LaTeX system  ***
% *** with the testflow diagnostic prior to trusting their LaTeX platform ***
% *** with production work. IEEE's font choices can trigger bugs that do  ***
% *** not appear when using other class files.                            ***
% The testflow support page is at:
% http://www.michaelshell.org/tex/testflow/


%%*************************************************************************
%% Legal Notice:
%% This code is offered as-is without any warranty either expressed or
%% implied; without even the implied warranty of MERCHANTABILITY or
%% FITNESS FOR A PARTICULAR PURPOSE! 
%% User assumes all risk.
%% In no event shall IEEE or any contributor to this code be liable for
%% any damages or losses, including, but not limited to, incidental,
%% consequential, or any other damages, resulting from the use or misuse
%% of any information contained here.
%%
%% All comments are the opinions of their respective authors and are not
%% necessarily endorsed by the IEEE.
%%
%% This work is distributed under the LaTeX Project Public License (LPPL)
%% ( http://www.latex-project.org/ ) version 1.3, and may be freely used,
%% distributed and modified. A copy of the LPPL, version 1.3, is included
%% in the base LaTeX documentation of all distributions of LaTeX released
%% 2003/12/01 or later.
%% Retain all contribution notices and credits.
%% ** Modified files should be clearly indicated as such, including  **
%% ** renaming them and changing author support contact information. **
%%
%% File list of work: IEEEtran.cls, IEEEtran_HOWTO.pdf, bare_adv.tex,
%%                    bare_conf.tex, bare_jrnl.tex, bare_jrnl_compsoc.tex
%%*************************************************************************

% Note that the a4paper option is mainly intended so that authors in
% countries using A4 can easily print to A4 and see how their papers will
% look in print - the typesetting of the document will not typically be
% affected with changes in paper size (but the bottom and side margins will).
% Use the testflow package mentioned above to verify correct handling of
% both paper sizes by the user's LaTeX system.
%
% Also note that the "draftcls" or "draftclsnofoot", not "draft", option
% should be used if it is desired that the figures are to be displayed in
% draft mode.
%
\documentclass[journal]{IEEEtran}

\usepackage[utf8]{inputenc}

% correct bad hyphenation here
\hyphenation{op-tical net-works semi-conduc-tor}


\begin{document}
%
% paper title
% can use linebreaks \\ within to get better formatting as desired
\title{Estimador de Estado de Sistemas de Distribuição de Energia Elétrica Utilizando Abordagem por Redes Neurais Artificiais com Modelo Reduzido}
%
%
% author names and IEEE memberships
% note positions of commas and nonbreaking spaces ( ~ ) LaTeX will not break
% a structure at a ~ so this keeps an author's name from being broken across
% two lines.
% use \thanks{} to gain access to the first footnote area
% a separate \thanks must be used for each paragraph as LaTeX2e's \thanks
% was not built to handle multiple paragraphs
%

\author{Luiz~Le~Roy,~\IEEEmembership{PUC Minas}
        % <-this % stops a space
\thanks{Texto finalizado em 15 de fevereiro de 2014.}}

% The paper headers
\markboth{Journal of \LaTeX\,~Vol.~X, No.~Y, Fevereiro~2014}%
{Shell \MakeLowercase{\textit{et al.}}: Bare Demo of IEEEtran.cls for Journals}

% make the title area
\maketitle

\begin{abstract}
%\boldmath
O artigo apresenta uma forma alternativa de modelagem para um estimador de estado de sistemas de distribuição (DSSE - Distribution System State Estimation). A partir de um determinado perfil de carga o estado real do sistema é gerado com auxílio de uma rede neural artificial (ANNs - Artificial Neuaral Networks).
O erro associado é utilizado para treinamento da rede através de minimização por mínimos quadrados ponderados (WLS - Weighted Least Squares). Utiliza-se a decomposição em diversos componentes através do ``modelo de mistura Gaussiano'' (GMM - Gaussian Mixture Model). No artigo do trabalho original, ``Distribution System State Estimation Using an Artificial Neural Network Approach for Pseudo Measurement Modeling'' de Manitsas et all (2012), o resultado do trabalho é demonstrado em um sistema de 95 barras do Reino Unido (U.K Generic Distribution System - UKGDS).
\end{abstract}

\begin{IEEEkeywords}
Sistemas de Distribuição de Energia Elétrica, DSSE, ANNs, WLS, GMM.
\end{IEEEkeywords}






% For peer review papers, you can put extra information on the cover
% page as needed:
% \ifCLASSOPTIONpeerreview
% \begin{center} \bfseries EDICS Category: 3-BBND \end{center}
% \fi
%
% For peerreview papers, this IEEEtran command inserts a page break and
% creates the second title. It will be ignored for other modes.
\IEEEpeerreviewmaketitle



\section{Introduction}
\IEEEPARstart{N}{este} texto, temos a principal referência o artigo ~\cite{IEEE-ann:principal}. É possível observar que o Estimador de Estado de Sistemas de Distribuição é uma ferramenta que providencia um satisfatório resultado a partir de um número limitado de medições.

\section{Conclusion}
The conclusion goes here.

%\appendices
%\section{Proof of the First Zonklar Equation}
%Appendix one text goes here.

% use section* for acknowledgement
\section*{Acknowledgment}
The authors would like to thank...


% Can use something like this to put references on a page
% by themselves when using endfloat and the captionsoff option.
\ifCLASSOPTIONcaptionsoff
  \newpage
\fi

\begin{thebibliography}{1}

\bibitem{IEEE-ann:principal}
E.~Manitsas, R. Singh, B. C. Pal, and G.~Strbac, \emph{``Distribution System State Estimation Using an Artificial Neural Network Approach for Pseudo Measurement Modeling,''},\hskip 1em plus
  0.5em minus 0.4em\relax IEEE Transaction on Power Systems, vol. 27, no. 4, pp. 1888-1896, Nov. 2012.

\end{thebibliography}

\begin{IEEEbiographynophoto}{Luiz Le Roy}
\'E um estudante do curso de mestrado da Pontif\'icia Universidade Cat\'olica de Minas Gerais.
\end{IEEEbiographynophoto}

% that's all folks
\end{document}
