\section{Conclus�es} \label{con}
Durante este trabalho o objetivo foi minimizar perdas el�tricas do circuito de distribui��o de energia e adequar o n�vel de tens�o ao longo das barras do alimentador. Optou-se por utilizar uma minimiza��o evolucion�ria, para encontrar pontos candidatos � instala��o de capacitores, com o objetivo de injetar pot�ncia reativa que atenda aos crit�rios mencionados.

O fator diferencial foi buscar a solu��o do problema em redes de distribui��o extensas. A abordagem escolhida permite a automatiza��o do processo e demanda menor interfer�ncia de um especialista na an�lise do problema. Os valores de tens�o descritos em normas foram atingidos para um circuito radial de 2498 barras. As perdas el�tricas, neste mesmo circuito, foram para patamares inferiores ap�s a inser��o de 9 bancos de capacitores trif�sicos de 150 kVAr.

Este tipo de abordagem capacita a concession�ria � alocar melhor os recursos dispon�veis. Principalmente quando o n�mero de barras e linhas do circuito � superior a uma centena de unidades. Nestes casos uma an�lise t�cnica ponto a ponto � praticamente invi�vel, e pode levar � m�nimos locais. Portanto, um processo automatizado, com uma confiabilidade aceit�vel, � vantajoso para solu��o do problema OCP.